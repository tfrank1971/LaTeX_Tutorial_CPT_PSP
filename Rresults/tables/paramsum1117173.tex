\begin{table}[ht]
\begin{threeparttable}
% Mon May 17 11:50:45 2021
\centering
\caption{Parameter estimates and standard errors from model no. 1117173} 
\label{tab:paramsum1117173}
\begingroup\small
\begin{tabular}{lccc}
  \hline
 & Estimate (\%RSE) & 95\% CI & Shrinkage(\%) \\ 
  \hline
Structural Model: &  &  &  \\ 
  TVCL (L/h)& 2.88 (3.1) & 2.7 - 3.06 &  \\ 
  TVV (L)& 32.3 (3.78) & 29.9 - 34.8 &  \\ 
  WTCLEXP & 0.75 (NA) & NA - NA &  \\ 
  WTVEXP & 1 (NA) & NA - NA &  \\ 
  AGECLEXP & -0.529 (18) & -0.72 - -0.339 &  \\ 
  SEXCLEXP & 1 (NA) & NA - NA &  \\ 
  SEXVEXP & 1 (NA) & NA - NA &  \\ 
  Inter-individual Variability ($\omega$): &  &  &  \\ 
  ETCL:ETCL & 0.185 (33) & 0.108 - 0.238 & 15 \\ 
  ETV:ETV & 0.225 (29.6) & 0.143 - 0.284 & 16.2 \\ 
  Residual Error ($\sigma$): &  &  &  \\ 
   (PERR:PERR) & 0.224 (14.8) & 0.187 - 0.255 & 17.5 \\ 
   \hline
\end{tabular}
\endgroup
\begin{tablenotes}\footnotesize
\item[] TVCL, clearance; TVV, volume of distribution; covariate relationships: WTCLEXP, CL \textasciitilde WT, WTVEXP, V \textasciitilde WT; AGECLEXP, CL \textasciitilde AGE; AGEVEXP, V \textasciitilde AGE; SEXCLEXP, CL \textasciitilde SEX; SEXVEXP, V \textasciitilde SEX
\item[] RSE = relative standard error, SD = standard deviation; SE = standard error; CI = confidence interval calculated as 95\%~CI = Point estimate $\pm 2 \cdot$ SE; NA = not applicable.
\item[] RSE of parameter estimate is calculated as 100 × (SE/typical value).
\item[] RSE of inter-individual variability magnitude is presented on \%CV scale and approximated as 100 × (SE/variance estimate)/2.
\item[] Shrinkage is calculated as 100 × (1 – SD of post hocs/$\omega$), with $\omega$ = sqrt(variance estimate).
\end{tablenotes}
\end{threeparttable}
\end{table}
