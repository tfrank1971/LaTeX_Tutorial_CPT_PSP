\begin{table}[ht]
\begin{threeparttable}
% Mon May 17 11:50:25 2021
\centering
\caption{Parameter estimates and standard errors from model no. 1117172} 
\label{tab:paramsum1117172}
\begingroup\small
\begin{tabular}{lccc}
  \hline
 & Estimate (\%RSE) & 95\% CI & Shrinkage(\%) \\ 
  \hline
Structural Model: &  &  &  \\ 
  TVCL (L/h)& 3.03 (3.83) & 2.8 - 3.26 &  \\ 
  TVV (L)& 32.4 (4.88) & 29.2 - 35.5 &  \\ 
  WTCLEXP & 0.66 (24.3) & 0.34 - 0.98 &  \\ 
  WTVEXP & 1.32 (15.3) & 0.918 - 1.72 &  \\ 
  AGECLEXP & -0.534 (19.3) & -0.74 - -0.328 &  \\ 
  AGEVEXP & 0.0523 (247) & -0.206 - 0.311 &  \\ 
  SEXCLEXP & 0.904 (5.68) & 0.801 - 1.01 &  \\ 
  SEXVEXP & 0.947 (7.13) & 0.812 - 1.08 &  \\ 
  Inter-individual Variability ($\omega$): &  &  &  \\ 
  ETCL:ETCL & 0.175 (29.8) & 0.111 - 0.221 & 16.1 \\ 
  ETV:ETV & 0.216 (29.7) & 0.138 - 0.273 & 17.1 \\ 
  Residual Error ($\sigma$): &  &  &  \\ 
   (PERR:PERR) & 0.224 (13.4) & 0.192 - 0.253 & 17 \\ 
   \hline
\end{tabular}
\endgroup
\begin{tablenotes}\footnotesize
\item[] TVCL, clearance; TVV, volume of distribution; covariate relationships: WTCLEXP, CL \textasciitilde WT, WTVEXP, V \textasciitilde WT; AGECLEXP, CL \textasciitilde AGE; AGEVEXP, V \textasciitilde AGE; SEXCLEXP, CL \textasciitilde SEX; SEXVEXP, V \textasciitilde SEX
\item[] RSE = relative standard error, SD = standard deviation; SE = standard error; CI = confidence interval calculated as 95\%~CI = Point estimate $\pm 2 \cdot$ SE; NA = not applicable.
\item[] RSE of parameter estimate is calculated as 100 × (SE/typical value).
\item[] RSE of inter-individual variability magnitude is presented on \%CV scale and approximated as 100 × (SE/variance estimate)/2.
\item[] Shrinkage is calculated as 100 × (1 – SD of post hocs/$\omega$), with $\omega$ = sqrt(variance estimate).
\end{tablenotes}
\end{threeparttable}
\end{table}
