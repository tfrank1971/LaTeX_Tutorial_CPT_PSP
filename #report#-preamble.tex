%==========================================================================
% definitions to be loaded just before the document environment
%==========================================================================
      
\usepackage{polyglossia}
    \setdefaultlanguage[variant=american]{english}
  
\usepackage[times]{fontsetup} %Loads the FreeSerifb fonts, a Times font and stix2 for Mathematics with letters replaced from FreeSerif   
    \setsansfont{Arial} %The main font-selecting command, which selects the standard fonts used in a document
    \setmonofont{Courier New} %The main font-selecting command, which selects the standard fonts used in a document
    
\usepackage{datetime2}
  \usepackage{setspace}
  \usepackage{ifthen}
  \usepackage{multicol}
  \usepackage{multirow}
  \usepackage{calc}
  \usepackage{units} 
  \usepackage{array,ragged2e,longtable}
  \usepackage{tabularx} 
  \usepackage{booktabs} 
  \usepackage{capt-of}
  \usepackage[automark]{scrlayer-scrpage} 
  \usepackage{lastpage} 
  \usepackage[multidot]{grffile}
  \usepackage{pdfpages} 
  \usepackage{graphicx}
  \usepackage{verbatim}
  \usepackage{moreverb}
  \usepackage[labelfont=bf,font=small]{caption}
  \usepackage{rotating}
  \usepackage{threeparttable, threeparttablex} % enables footnotes under tables
  \usepackage{pdflscape} % rotating pages in landscape format 
  \usepackage[addtotoc]{abstract} % include summary in ToC, http://www.tug.org/texlive/Contents/live/texmf-dist/doc/latex/abstract/abstract.pdf
  \usepackage{textcomp} % kile special characters
  \usepackage{xcolor} % colored text
  \usepackage{listings} % for listings in Appendix
  \usepackage{listingsutf8} % extension in case there are umlauts
  \usepackage{appendix} % prints APPENDIX heading in ToC
  \usepackage{subcaption} % replaces subfig and subfigure
  %\usepackage{cite} % correct refs for multiple citation, incompatible with biblatex 
    \usepackage{enumitem}
  \usepackage{makeidx}
  \usepackage[autostyle=true]{csquotes}
  \usepackage[backend=biber,style=nejm, maxbibnames=6, minbibnames=6]{biblatex} %biblatex numeric style based on the design of the New England Journal of Medicine (NEJM). 
   
  \addbibresource{#report#.bib} % replaces old \bibliography command
%\bibliography{publications}      % name of bib-file without .bib
 %\setlength{\bibitemsep}{1em}     % distance between items
 %\setlength{\bibhang}{2em}        % indentation after first line
  \usepackage{microtype} % better typography, {babel} should come before
  \usepackage[toc,acronym,nonumberlist,nogroupskip,nopostdot,order=letter]{glossaries} 

  %\makeindex 
  \setacronymstyle{long-short}
  \makeglossaries
 
 \loadglsentries{myacronyms}  % TeX-file containing the acronyms

%-------------------------------------------------------------------------
  
  \setcounter{secnumdepth}{4} % adapt depth of numbering in ToC
  \setcounter{tocdepth}{4} % adapt depth of numbering in ToC
  %\renewcommand{\abstractname}{SUMMARY} % new name for abstract
  
% this is for lazy editors
  \newcommand{\fs}{\footnotesize}
  \newcommand{\scs}{\scriptsize}  


% paragraph control #############################################################

\clubpenalty=10000  %prevent orphan line
\widowpenalty=10000 %prevent widow lines

%==========================================================================
  % macros used for plotting
%==========================================================================


  \newcommand{\scale}{2.5}
  \newcommand{\plotscale}{1} %zooms the figure to x% of its original size (default 1 = 100%)

  \newcommand{\plot}[4] { %\parskip0pt
        \begin{figure}[ht]
          \caption[#1]{#4}
          \centering{\includegraphics[angle=\angle,width=\plotscale\textwidth]{#2}}
          \label{#3}
        \end{figure}
        }

  \newcommand{\plotB}[5] { %\parskip0pt
        \begin{figure}[ht]
          \caption[#1]{#5}
          \centering{\includegraphics[angle=\angle,width=\plotscale\textwidth]{#2}}
          \centering{\includegraphics[angle=\angle,width=\plotscale\textwidth]{#3}}
          \label{#4}
        \end{figure}
        }

  \newcommand{\plotC}[6] { %\parskip0pt
        \begin{figure}[ht]
          \caption[#1]{#6}
          \centering{\includegraphics[angle=\angle,width=\plotscale\textwidth]{#2}}
          \centering{\includegraphics[angle=\angle,width=\plotscale\textwidth]{#3}}
          \centering{\includegraphics[angle=\angle,width=\plotscale\textwidth]{#4}}
          \label{#5}
        \end{figure}
        }

  \newcommand{\lplot}[4] { %\parskip0pt
        \begin{sidewaysfigure}[h]
          \caption[#1]{#4}
          \centering{\includegraphics[angle=\angle,width=\plotscale\textwidth]{#2}}
          \label{#3}
        \end{sidewaysfigure}
        }

  \newcommand{\lplotB}[5] { %\parskip0pt
        \begin{sidewaysfigure}[h]
          \caption[#1]{#5}
          \centering{\includegraphics[angle=\angle,width=\plotscale\textwidth]{#2}}
          \centering{\includegraphics[angle=\angle,width=\plotscale\textwidth]{#3}}
          \label{#4}
        \end{sidewaysfigure}
        }

  \newcommand{\lplotC}[6] { %\parskip0pt
        \begin{sidewaysfigure}[h]
          \caption[#1]{#6}
          \centering{\includegraphics[angle=\angle,width=\plotscale\textwidth]{#2}}
          \centering{\includegraphics[angle=\angle,width=\plotscale\textwidth]{#3}}
          \centering{\includegraphics[angle=\angle,width=\plotscale\textwidth]{#4}}
          \label{#5}
        \end{sidewaysfigure}
        }

  \newcommand{\spllistref}[1] {
        The code for the S+ function for the call: \input{plots/#1} is found
        in Listing \ref{#1} on page \pageref{#1}. 
        }

% =========== Intext Plots=================================================

   \newcommand{\plotIntext}[4] { %\parskip0pt
        \begin{figure}[htbp]
          \caption[#1]{#4}
          \centering{\includegraphics[angle=\angle,width=\plotscale\textwidth]{#2}}
          \label{#3}
        \end{figure}
        }

  \newcommand{\plotBIntext}[5] { %\parskip0pt
        \begin{figure}[htbp]
          \caption[#1]{#5}
          \centering{\includegraphics[angle=\angle,width=\plotscale\textwidth]{#2}}
          \centering{\includegraphics[angle=\angle,width=\plotscale\textwidth]{#3}}
          \label{#4}
        \end{figure}
        }
        
     \newcommand{\plotIntextPNG}[4] { %\parskip0pt
        \begin{figure}[!htbp]
          \caption[#1]{#4}
          \centering{\includegraphics[type=png,read=.png,ext=.png,angle=\angle,width=\plotscale\textwidth]{#2}}
          \label{#3}
        \end{figure}
        }

% =========== Plots continuing on next page ==============================

  \newcommand{\plotCont}[4] { %\parskip0pt
        \begin{figure}[ht]\ContinuedFloat
          \caption[#1]{#4}
          \centering{\includegraphics[angle=\angle,width=\plotscale\textwidth]{#2}}
          \label{#3}
        \end{figure}
        }

  \newcommand{\plotBCont}[5] { %\parskip0pt
        \begin{figure}[ht]\ContinuedFloat
          \caption[#1]{#5}
          \centering{\includegraphics[angle=\angle,width=\plotscale\textwidth]{#2}}
          \centering{\includegraphics[angle=\angle,width=\plotscale\textwidth]{#3}}
          \label{#4}
        \end{figure}
        }

%====== cuts single pages out of a multiple pages pdf ====================
       
      \newcommand{\page}{1}
  \newcommand{\cuttedplot}[4] { %\parskip0pt
        \begin{figure}[htpb]
          \caption[#1]{#4}
          \centering{\includegraphics[page=\page, angle=\angle,width=\plotscale\textwidth]{#2}}
          \label{#3}
        \end{figure}
        }      
      
%==========================================================================
  % macros used for tables
%==========================================================================

 \newcommand{\dtable}[4]{
  	\begin{table}[!htbp]
\centering
		\caption[#1]{#2}
          \input{#3}\label{tab:#4}
	\end{table}
	}


\newcommand{\comptable}[2]{
\begin{threeparttable}
\input{#1}\vspace{-1cm}
\begin{tablenotes}\footnotesize 
         \item[] #2
     \end{tablenotes}%
\end{threeparttable}
}

	

  % control the width of tab within the verbatimtabinput environment
  \renewcommand\verbatimtabsize{3\relax}
%==========================================================================
%                           style definitions 
%==========================================================================

  %-------------begin-GENERAL LAYOUT-----------

\pagestyle{scrheadings}

\addtokomafont{disposition}{\sffamily}
\addtokomafont{pageheadfoot}{\normalfont \footnotesize \sffamily}
\addtokomafont{pagenumber}{\footnotesize \sffamily}

\clearpairofpagestyles

 \lohead{\vbox{\footnotesize{\DOC}  \\ {\heada}}}
 %\rohead{\vbox{{\headc} \\ \footnotesize{\DOC}}}
 \lofoot{\footnotesize{Property of <Institution> -- strictly confidential}}
 \rofoot[\pagemark]{\footnotesize{Page} \pagemark} % page number only outer right
%\chead{foobar} % name in the center
\setlength{\headheight}{26.35004pt} % Customize the height of the header

%-------------end-GENERAL LAYOUT-----------


\endinput
